\documentclass{article}

\usepackage{amsmath}
\usepackage[utf8]{inputenc}
\usepackage{graphicx}
\usepackage[portuguese]{babel}

\begin{document}

\section*{2}

A aproximação função $seno(x)$ no intervalo $[-\frac{\pi}{2}, 3 \frac{\pi}{2}]$
realizada por meio do modelo de Sugeno usa três regras \emph{fuzzy} e portanto
três antecedentes e consequentes. Os consequentes são lineares e os
antecedentes, funções de pertinência triangulares, são lineares por partes. O
desenvolvimento analítico a seguir se baseia na escolha de consequentes e
antecedentes ilustrada na figura \ref{fig:choice}. Observa-se que é possível
dividir a região de inferência entre $[\frac{-\pi}{2}, \frac{\pi}{2}]$ e
$[\frac{\pi}{2}, 3 \frac{\pi}{2}]$ de forma que as funções de pertinência sejam
lineares.

\begin{figure}
    \centering
    \includegraphics[scale=0.25]{"choice.png"}
    \caption{Representação visual de espaço de inferência}
    \label{fig:choice}
\end{figure}

\subsection*{Consequentes}

Os consequentes foram definidos por inspeção da figura ~\ref{fig:choice} pelas
seguintes retas:

$y_1(x) = \frac{2}{\pi}x + 1$  \\
$y_2(x) = \frac{-2}{\pi}x + 1$ \\
$y_3(x) = \frac{2}{\pi}x - 3$ \\

\subsection*{Antecedentes}

As retas antecedentes das regiões 1 e 2 são definidas por inspeção da figura
~\ref{fig:choice}.

Observa-se que os antecedentes podem ser formalmente definidos como:

\subsubsection*{Região 1:[$-\frac{\pi}{2}, \frac{\pi}{2}$]}

$a_1(x) = \frac{-1}{\pi}x + \frac{1}{2}$ \\
$a_2(x) = \frac{1}{\pi}x + \frac{1}{2}$ \\

\subsubsection*{Região 2:[$\frac{\pi}{2}, \frac{3\pi}{2}]$}

$a_3(x) = \frac{-1}{\pi}x + \frac{1}{2} + \pi$ \\
$a_4(x) = \frac{1}{\pi}x + \frac{1}{2} + \pi$ \\

\subsection*{Inferência}

A inferência analítica é dada por partes:

\[ 
    \hat{y}(x) =
    \begin{cases}
    \frac{a_{1}(x)y_{1}(x) + a_{2}(x)y_{2}(x)}{a_{1}(x)+a_{2}(x)} & x \in [\frac{-\pi}{2}, \frac{\pi}{2}]\\
    \frac{a_{3}(x)y_{2}(x) + a_{4}(x)y_{3}(x)}{a_{3}(x)+a_{4}(x)} & x \in [\frac{\pi}{2}, \frac{3\pi}{2}] \\
   \end{cases}
\]

\end{document}
