\documentclass{article}

\usepackage[utf8]{inputenc}
\usepackage[portuguese]{babel}

\begin{document}

\section{2}

A aproximação função $seno(x)$ no intervalo $[-\pi/2, 3\pi/2\$
realizada por meio do modelo de Sugeno usa três regras \emph{fuzzy} e portanto
três antecedentes e consequentes. Os consequentes são lineares e os antecedentes,
funções de pertinência triangulares, são lineares por partes. O desenvolvimento
analítico a seguir se baseia na escolha de consequentes e antecedentes ilustrada
na figura \ref{fig:choice}. Observa-se que é possível dividir a região de
inferência entre $[-\pi/2, 0]$ e $[0, 3\pi/2\$ de forma que as funções de
pertinência sejam lineares.

\begin{figure}
    \centering
    \includegraphics{"choice.png"}
    \caption{Representação visual de espaço de inferência}
    \label{fig:choice}
\end{figure}

\subsection{Consequentes}

Definição das 3 retas consequentes

\subsection{Antecedentes}

Definição das 4 retas antecedentes

\subsubsection{Região 1}

Na regiao 1

\subsubsection{Região 2}

Na regiao 2

\subsection{Inferência}

expressão analítica para inferência (por partes)

\end{document}
